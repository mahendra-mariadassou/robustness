\documentclass[a4paper,12pt]{article}
\usepackage[utf8]{inputenc}

% Mise en page
\textwidth 16cm
\textheight 24cm
\topmargin -1 cm
\oddsidemargin 0cm
\evensidemargin 0cm

\usepackage{amsmath,amssymb,amsthm}
\usepackage{mathabx}
\usepackage{xcolor}
\usepackage[round]{natbib}
\usepackage{enumerate}
\usepackage{hyperref}
\usepackage{array}
\usepackage{graphicx}

% custom colors
\newcommand{\noir}[1]{\textcolor[gray]{0}{#1}}   %% black
\newcommand{\gris}[1]{\textcolor[gray]{0.5}{#1}}
\newcommand{\gray}[1]{\textcolor[gray]{0}{\text{#1}}}
\newcommand{\grey}[1]{\textcolor[gray]{0.5}{\text{#1}}} %% light-gray


%opening
\title{Tree Evaluation and Robustness Testing}
\author{C'est nous}

%%%%%%%%%%%%%%%%%%%%%%%%%%%%%%%%%%%%%%%%%%%%%%%%%%%%%%%%%%%%%%%%%%%%%%
%%%%%%%%%%%%%%%%%%%%%%%%%%%%%%%%%%%%%%%%%%%%%%%%%%%%%%%%%%%%%%%%%%%%%
\begin{document}
%%%%%%%%%%%%%%%%%%%%%%%%%%%%%%%%%%%%%%%%%%%%%%%%%%%%%%%%%%%%%%%%%%%%%%
%%%%%%%%%%%%%%%%%%%%%%%%%%%%%%%%%%%%%%%%%%%%%%%%%%%%%%%%%%%%%%%%%%%%%%

\date{\today}
\maketitle

\begin{abstract}
Inferring evolutionary histories (phylogenetic trees) has important applications but it is rare to observe a single evolutionary history unanimously implied by all genetic loci, and different genetic sites may conflict with respect to the implied phylogeny.  However, phylogenetic trees are complex mathematical objects that allows the study of the variability and bias of the estimates. In this paper we review the source of errors, how to construct consensus trees, outliers detection either at loci or species level as well as the definition on continuous distances on a tree space.
\end{abstract}

\tableofcontents

%%%%%%%%%%%%%%%%%%%%%%%%%%%%
%%%%%%%%%%%%%%%%%%%%%%%%%%%%
\section{Motivation} \label{sec:Motivation}

%%%%%%%%%%%%%%%%%%%%%%%%%%%%
\subsection{Applications of Phylogenies} \label{sec:applications}

Molecular phylogenetics is a lively field of research with a number of practical applications.
Reconstructing large phylogenies, such as the bird [ref] or mammal phylogeny [ref], is of intrinsic interest
to evolutionary biologists, but those phylogenies are also \emph{the basic structures necessary to think clearly about differences between species, and to analyze those differences statistically}~\citep{Felsenstein2004}. They arise frequently in comparative genomics, conservation issues~\citep{Bordewich2008}, functional prediction of genes~\citep{Eisen1998} and more generally are at the heart of phylogenetic comparative methods~\citep{Revell2008, Pennell2013}. Most, if not all, applications of phylogenetics have in common that they rely on accurate phylogenetic trees and it is crucial to validate the tree as different trees can lead to vastly different conclusions concerning the origin and evolution of a trait \citep{Geneva2015}.

With the advent of molecular data and increased formalism of the field~\citep{Gascuel2005a}, modern phylogenetic reconstruction is now essentially a statistical inference problem. Many popular reconstruction softwares such as PhyML~\citep{Guindon2003}, RAxML~\citep{Stamatakis2006}, FastTree~\citep{Price2010} or MrBayes~\citep{Ronquist2003} produce a statistical estimate of the tree and we therefore frame validation in a statistical framework. We first discuss the different sources of inaccuracies in the reconstructed tree (Section~\ref{sec:error-sources}) and distinguish between natural variability ($\simeq$ variance) and modeling errors ($\simeq$ bias). We then briefly describe and discuss popular support values (Section~\ref{sec:robustness}) aimed at validating a tree. In addition to support values, the variability observed in a forest of trees, can be summarized in order to produce robust tree estimates (Section~\ref{sec:consensus}). Finally, we end this chapter by reviewing promising developments in the field of continuous distance and highlight their use for validation (Section~\ref{sec:extensions})

% We distinguish to natural uncertainty, due to lack of signal in the data, and other sources of inaccuracies, arising for example from erroneous sequence alignments, misspecified evolution models, etc.

%%%%%%%%%%%%%%%%%%%%%%%%%%%%
\subsection{Validating the Tree} \label{sec:tree-validation}

A tree is a complex object that encodes the evolutionary relationship of a set of species. Many inference methods return a single focal tree and validation is most often concerned with it. There are two families of validation methods based on (i) the scale at which validation is performed and (ii) whether the tree is considered by itself or with respect to other trees.

\begin{itemize}
 \item The first family is \textbf{local} and grounded on the observation that a tree is uniquely determined by its branches \citep{Buneman1971}. A tree can thus be validated by computing a \emph{support value} for each of its branch. Support values tell us which parts of the tree are \emph{reliable}, in a yet to be defined way.
 \item The second family is \textbf{global} and considers the focal tree as a single object rather than a collection of branches. It compares it to a set of alternative trees using statistical tests \citep{Shimodaira1999}. The tests tell us whether the tree is strictly better (\emph{i.e.} a better fit to the molecular data) than the alternatives.
\end{itemize}

The two approaches have a different focus but are complimentary. In particular, global tests can be used to compute local support values.

%%%%%%%%%%%%%%%%%%%%%%%%%%%%
\subsection{Robust Estimate} \label{sec:robust-estimate}

Most softwares return the best tree for a given criteria (\emph{e.g.} likelihood for PhyML and RAxML). Support values tell us whether that tree is reliable and tests tell us whether it's much better than second best or other alternatives.

However, in the presence of outlier data, the superiority of the best tree may lie exclusively in a few data points: slight changes in the molecular data may dramatically change the best tree, with deep clades moving from one position in the tree to another~\citep{Bar-Hen2008}. Since molecular data are inherently noisy, it is interesting to produce \textbf{robust} trees that nearly, but not completely, optimize the criteria while being resilient to small changes in the molecular data.

A straighforward way to build a robust estimate is to start from a forest of \emph{good} trees and summarize them in some way to build a \textbf{consensus} tree. The forest can consist of trees that are only slightly worse than the best tree (\emph{e.g} bayesian consensus) or that are inferred from slightly perturbed data (\emph{e.g.} bootstrap consensus). 

However,  more  recently,  the  statistical perspective that observed variability between collections of trees is interesting of itself has gained traction. This has spawned a number of modern methods that utilize the elegant mathematics of tree space to permit new data analysis methods and associated new insights in a number of different contexts

%%%%%%%%%%%%%%%%%%%%%%%%%%%%
%%%%%%%%%%%%%%%%%%%%%%%%%%%%
\section{Sources of Error} \label{sec:error-sources}

%We do not address the species tree/gene tree discrepancy.
%
%\begin{itemize}
% \item Sampling errors
% \item Modeling errors
%\end{itemize}
Genome-scale analyses indicate that different genes yield conflicting phylogenies. Broadly speaking, these conflicts arise from two main sources: (i) the inability of traditional reconstruction methods to deal with the complexity of molecular evolution (methodological sources) and/or (ii) genuine biological events such as lateral gene transfer (LGT), incomplete lineage sorting and others that lead to different evolutionary histories along the genome (biological sources).

Methodological factors affecting phylogenetic reconstruction include the choice of optimality criterion, limited data availability, taxon sampling and specific assumptions in the modelling of sequence evolution. Biological processes such as the natural selection, small population size, etc may also cause the the gene tree to differ from the species tree. The large number of potential explanations for observed incongruences in molecular phylogenetics makes decisions of how to handle them quite difficult \citep{rokas2003genome}.

%%%%%%%%%%%%%%%%%%%%%%%%%%%%
\subsection{Biological Sources} \label{sec:sampling-error}
%Noisy data (imperfect alignments, sequencing error, one individual per species, different parts of the gene/genome can have different evolutionary histories).

Biologial source of errors are very diverse and among other we may cite  contamination,  frameshift  events,  incorrect  annotations,  erroneous  chimerical  sequences,  wrong orthology assessment, horizontal gene transfer, gene conversion, incomplete lineage sorting or hybridization, etc. (see \cite{philippe2017pitfalls} for a survey)

With the improvement of sequencing technologies, the amount and quality of molecular data used in phylogenetics has drastically increased to the point where sequence quality is superseded by other concerns. For example, inclusion of non-orthologous sequences in a study can have drastic consequences on the final results \citep{laurin2012origin,philippe2011resolving}. It can for example arise through undectected contamination of the genetic material during the sampling or sequencing steps (cross-contamination). It can also arise when paralogs are misidentified as orthologs, a common occurence given the high frequency of gene/genome duplication, gene conversion and gene loss. 

% Due to the limitations of ancient sequencing technologies, sequencing errors were frequent and sequence quality was quite variable in these early datasets. High-throughput  sequencing  has  greatly  improved  sequence  quality,  mainly  through  the  large  coverage  of each nucleotide, but has simultaneously flooded researchers with an amount of data that is  impossible to  handle  by  hand. The most frequent errors observed are sequencing errors (especially for transcriptomic data) and annotation errors (especially for genomic data).  Errors due to the inclusion of non-orthologous sequences in phylogenomics can have drastic consequences on the final results (\cite{laurin2012origin,philippe2011resolving}).  Contamination is not the only source of non-orthology, paralogy being a common underhand source of issues, given the high frequency of gene/genome duplication, gene conversion and gene loss. Sequence contamination can occurr at the sampling step or at the laboratory or sequencing steps (cross-contamination).

Finally, correct alignments are of crucial importance in phylogenetics as repeteadly pointed out \citep{morrison1997effects,ogden2006multiple,talavera2007improvement,wong2008alignment}. Yet, due to the lack of tractable models of sequence evolution in the presence of insertion and deletion events (indels), the criteria optimized by alignment software are mostly ad hoc  and based on the simplistic assumptions that homologous characters should be similar and that indels are rare events.

%%%%%%%%%%%%%%%%%%%%%%%%%%%%
%\subsection{Limited Amount of Informations} \label{sec:small-n}
%Noisy data (imperfect alignments, sequencing error, one individual per species, different parts of the gene/genome can have different evolutionary histories).

%%%%%%%%%%%%%%%%%%%%%%%%%%%%
\subsection{Modeling Errors} \label{sec:modeling-errors}
%Site independence
%
Every model is only a rough sketch of the complex reality of molecular evolution. First and foremost, the assumption of independent and identical evolutionary forces across sites is certainly not true in reality \citep{Piau}. What happens at one site depends both on its genomic context and on its interaction with other sites in the secondary and ternary structures of the molecule. Some relaxation allow sites to evolve at different sites (\cite{goldman1994codon}). Formally, each site has its own rate, modeled as a random effect drawn from a gamma distribution. \cite{felsenstein1996hidden} extended this approach to allow the rate at one site to depend on neighbouring sites rates using a hidden Markov model. This model captures dependence on the local genomic context (linear, primary structure) but unfortunately not dependence induced by proximity of sites in the secondaty and ternay structure.

For protein evolution, the rate matrices used to model evolution (\emph{e.g} JTT, PAM, LG, etc) are derived by averaging over patterns observed in thousands of sites. However it is clear \citep{halpern1998evolutionary,parisi2001structural,susko2002testing} that observed amino acid frequencies strongly deviate from the one expected under the JTT matrix. \citet{lartillot2004bayesian} implemented a Bayesian mixture model which allows the inference of site specific rate matrices. These studies show that relaxing the assumption that all sites evolve according to the same rate matrices is crucial for accurate phylogenetic inference. However, such parameter-rich pose problems of their own, which can be difficult to diagnose (see \citet{rannala2002identifiability} for a  discussion of over-parameterization in the context of Bayesian phylogenetic inference).

Additionnally, many models  assume stationarity, homogeneity and reversibility of sequence evolution. This is at odds with observations in archea, where evolution exhibit a trend towards some nucleotides over very long time scales \citep{boussau2006efficient}. In bacteria, codon usage bias (ref codon usage bias) can lead to non reversible evolution and different underlying stationarity state frequencies in different parts of the tree. If distantly related lineages begin to display similar state frequencies, these lineages can be artefactually grouped together when using evolution model that do not explicity allow it (see for example \cite{foster2004modeling,jermiin2004biasing}). Finally, phylogenetic trees are probably not the adequate tool to analyse and represent non-vertical transmission and reticulated evolution \citep{huson2005application}.

% non-vertical transmission and more generally non-reticulated evolution is best captured by phylogenetic networks rather than phylogenetic trees \citep{huson2005application}. 
% For instance, the set of nucleotide or amino acid positions free to vary, and the evolutionary rates at which they vary, and the frequencies of nucleotides, codons and amino acids can periodically change over the tree, due to functional or structural alterations in the molecule or differing selective forces in the organisms' genomes.

% Other cases where the homogeneity, stationarity and/or time-reversibility of substitution models are violated include situations where the underlying state frequencies (whether they be nucleotides, amino acids or codons) in the genes of different organisms vary over the tree. 

%%%%%%%%%%%%%%%%%%%%%%%%%%%%
% \subsubsection{Complex Evolutionary History} \label{sec:complex-history}

% Incomplete Lineage Sorting (in a multi tree Context)  %%1 page %% ABH

%%%%%%%%%%%%%%%%%%%%%%%%%%%%
%%%%%%%%%%%%%%%%%%%%%%%%%%%%
\section{Robustness} \label{sec:robustness}

As discussed in Section~\ref{sec:Motivation}, support values are a popular way to validate a focal tree. We present here the most popular ones before describing other methods to validate a tree or fortify it. 

%%%%%%%%%%%%%%%%%%%%%%%%%%%%
\subsection{Support Values} \label{sec:confidence-values}

%%%%%%%%%%%%%%%%%%%%%%%%%%%%
\subsubsection{Bootstrap} \label{sec:bootstrap}

Bootstrap values~\cite{Felsenstein1985} are probably the most popular and easiest to understand support values. Bootstrap involves resampling with replacement from one's molecular data with to create fictional datasets, called \emph{bootstrap replicates}, of the same size. Specifically, the molecular data is typically organized as a multiple sequence alignment (MSA) of $s$ species $\times$ $n$ characters. Since most models assume independent characters, we generate a replicate by sampling $n$ characters, with replacement, from the original MSA and do this $B$ times. Note that in each replicate, some characters are sampled more than once and some left out entirely. The $B$ replicates are used to estimate a forest of $B$ bootstrap trees (one per replicate). Finally the bootstrap value ($BP$) of a branch of the original tree is its frequency of occurrence in the forest. The process is illustrated in Figure~\ref{fig:bootstrap}. 

\begin{figure}
  \begin{center}
	\begin{tabular}{>{\centering\arraybackslash}m{4cm}cc}
	  \textbf{MSA} & & \textbf{Inferred Tree} \\
	  %% Original data
	  & \\
	  \multicolumn{3}{c}{Original Data}\\
	  $
	  \begin{array}{|c|cccc|}
	    \hline
	    \textbf{A} & \noir{A} & \gris{C} & \gray{T} & \grey{T} \\
	    \textbf{B} & \noir{G} & \gris{G} & \gray{A} & \grey{T} \\
	    \textbf{C} & \noir{G} & \gris{G} & \gray{C} & \grey{C} \\
	    \hline
	  \end{array}
	  $
	  &
	  $\longrightarrow$
	  &
	  \begin{minipage}[c]{0.25\linewidth}
	    \begin{center}
	      \includegraphics[width=0.6\linewidth]{Figs/TrueOne3.pdf}
	    \end{center}
	  \end{minipage} 
	  \\
	  %% Replicate 1
	  & \\
	  \multicolumn{3}{c}{Bootstrap Replicate \#1}\\
	  $
	  \begin{array}{|c|cccc|}
	    \hline
	    \textbf{A} & \noir{A} & \gris{C} & \gray{T} & \gris{C} \\
	    \textbf{B} & \noir{G} & \gris{G} & \gray{A} & \gris{G} \\
	    \textbf{C} & \noir{G} & \gris{G} & \gray{C} & \gris{G} \\
	    \hline
	  \end{array}
	  $
	  &
	  $\longrightarrow$
	  &
	  \begin{minipage}[c]{0.25\linewidth}
	    \begin{center}
	      \includegraphics[width=0.6\linewidth]{Figs/TrueOne1.pdf}
	    \end{center}
	  \end{minipage}
	  \\
          %% Replicate 2
	  & \\
	  \multicolumn{3}{c}{Bootstrap Replicate \#2}\\
	  $
          \begin{array}{|c|cccc|}
            \hline
            \textbf{A} & \gris{C} & \noir{A} & \gray{T} & \noir{A} \\
            \textbf{B} & \gris{G} & \noir{G} & \gray{A} & \noir{G} \\
            \textbf{C} & \gris{G} & \noir{G} & \gray{C} & \noir{G} \\
            \hline
          \end{array}
          $
          &
          $\longrightarrow$
          &
          \begin{minipage}[c]{0.25\linewidth}
            \begin{center}
              \includegraphics[width=0.6\linewidth]{Figs/TrueOne1.pdf}
            \end{center}
          \end{minipage}
          \\
          %% Replicate 3
	  & \\
	  \multicolumn{3}{c}{Bootstrap Replicate \#3}\\
	  $
          \begin{array}{|c|cccc|}
            \hline
            \textbf{A} & \gray{T} & \grey{T} & \grey{T} & \gray{T} \\
            \textbf{B} & \gray{A} & \grey{T} & \grey{T} & \gray{A} \\
            \textbf{C} & \gray{C} & \grey{C} & \grey{C} & \gray{C} \\
            \hline
          \end{array}
	  $
          &
          $\longrightarrow$
          &
          \begin{minipage}[c]{0.25\linewidth}
            \begin{center}
              \includegraphics[width=0.6\linewidth]{Figs/TrueOne2.pdf}
            \end{center}
          \end{minipage}
          \\
	\end{tabular}
   \end{center}
   \caption{Principle of the bootstrap for phylogenies. Each character is identified by its color and style. Characters are sampled with replacement to produce bootstrap replicates, which are then used to infer phylogenies. The split $A|BC$ appears in $2$ out of $3$ bootstrap trees and therefore has a bootstrap value of $BP = 2/3$ or $66$\%.}
  \label{fig:bootstrap}
\end{figure}

Intuitively, the variation obtained by resampling $n$ sites from the original data should be the same as the variation obtained by sampling $n$ new characters. Bootstrap values capture, among other, the \emph{sampling} variability induced by short MSA. When $n$ increases, so does $BP$ in general and it is quite common to achieve very high values for all branches when working on genome-scale alignments~\citep{Rokas2003}. 

$BP$ provides a guide for the amount of support a branch has: branches with high $BP$ occur more often and are more reliable than those with low $BP$. Although it might be tempting to interpret $BP$ as the probability that a branch is present in the (unknown) true tree, this is not the case in general. \cite{Zharkikh1992} showed in a simple case that $BP$ is biased and underestimates that probability. Using simulation studies, \cite{Hillis1993} showed that $BP$ values as small as 70\% could reflect highly supported branches. Many studies~\citep{Felsenstein1993, Efron1996, Susko2008, Susko2010} examined the theoretical properties of bootstrap values and concluded that they are indeed biased. This bias is partly induced by the peculiar geometry of tree space (see \cite{Billera2001, Susko2010} and Section~\ref{sec:extensions}). 

The final limitation of bootstrap values, shared with other support values based on resampling techniques, is their high computational cost: the budget required to compute $B$ bootstrap trees is $B$-times higher than the budget for the original tree. Clever implementations can substantially reduce that cost~\citep{Stamatakis2014} but it remains prohibitive for very large trees. 

%%%%%%%%%%%%%%%%%%%%%%%%%%%%
\subsubsection{Posterior Probabilities} \label{sec:posterior-probabilities}

Posterior Probabilities ($PP$) are mostly used in a Bayesian framework and similar in spirit bootstrap values. The main difference lies in the forest of trees used to compute support values. Bayesian procedures estimate the posterior distribution of trees. In practice, the distribution is too complex to fully explore and software produce a Monte Carlo Markov Chain (MCMC) sample from the posterior distribution~\citep{Yang1997a}. The $PP$ of a branch is computed, just like $BP$, as the the probability of occurrence of that branch in the MCMC sample. MCMC trees constitute a set of highly likely trees for the original dataset. $PP$ are easier to interpret than $BP$ as they approximate directly the probability that a branch is present in the true tree, given the original data. Furthermore, since MCMC trees are a natural byproduct of the bayesian estimation procedure, there is almost no overhead in computing $PP$. 

Unfortunately, $PP$ are not immune to bias. Empirical studies found that $PP$ are generally higher than $BP$~\citep{Anisimova2011} and sometimes even overconfident, with the ``star-tree paradox'' \citep{Yang2007} being the perfect example of overconfidence. \citet{Yang2007} showed that when the actual tree is a 3-species star tree, so that all $3$ potential inner branches are wrong, the bayesian method picks at random whose $PP$ goes to $100$\% when sequence length goes to infinity, whereas one could expect the $PP$ to fluctuate around $33$\%. 

Intuitively, $PP$ are higher than $BP$ because they cover fewer sources of variability. Unlike bootstrap trees, MCMC trees all originate from the same dataset. $PP$ are quite good at capturing the lack of phylogenetic signal in the original MSA but not the impact of a few influential characters. For example, outlier characters with a strong effect on the tree or a tiny majority of character that favor one inner branch over another will affect all MCMC trees consistently. By contrast, they will be included in some bootstrap replicates but left out from others leading to more variation among bootstrap trees than among MCMC trees. Finally, in genome-scale context where inaccuracies are more likely to arise from modeling errors than from sampling variability, $PP$ are uniformly high and as uninformative as $BP$~\citep{Philippe2011, Kumar2012}.

%%%%%%%%%%%%%%%%%%%%%%%%%%%%
\subsubsection{Likelihood-based Support Values} \label{sec:other-confidence}

Both $BP$ and $PP$ quantify the agreement between a focal tree and forest of trees. Likelihood-based supports are fast alternatives that bypass the need for a forest and deal exclusively with the focal tree~\citep{Anisimova2006}. 

For any inner branch in the focal tree, there are $3$ NNI configurations around that branch: the focal one $T_1$ and two alternatives $T_2$, $T_3$ (see Figure~\ref{fig:nni}). If we note $\ell_i = \log Pr(D \| T_i)$ the likelihood of the data under tree $i$ and assume that $T_1$ is the maximum-likelihood tree, we have $\ell_1 \geq \max(\ell_2, \ell_3)$. Likelihood-based supports values essentially test whether $\delta = \ell_1 - \max(\ell_2, \ell_3)$ is significantly larger than $0$. 

\begin{figure}
 \includegraphics[width=0.9\linewidth]{Figs/NNI}
 \caption{The maximum likelihood tree ($T_1$, left) and its two NNI-alternatives ($T_{2}$ middle and $T_{3}$ right) corresponding to different resolutions of the inner branch. Subtrees are sketched as triangles.}
 \label{fig:nni}
\end{figure}


The most popular support values are:
\begin{itemize}
 \item the approximate Likelihood Ratio Tests (aLRT) values which evaluates the statistics $\delta$ and compares it to $0.5\chi^2_0 + 0.5\chi^2_1$ to compute a p-value. The p-value is then converted into a support value between $1/8$ and $1$. A branch with high $\delta$ will have high support. 
 \item the SH-corrected aLRT (SH-aLRT) values are based on the same idea but use the non-parametric~\citet{Shimodaira1999} procedure to compute the p-value of $\delta$.
 \item Finally approximate bayes (aBayes) is an approximation of the posterior probability of tree $T_i$ computed as:
 \[
  Pr(T_i | D) = \frac{Pr(T_i) Pr(D | T_i)}{\sum_{j=1}^3 Pr(T_j) Pr(D | T_j)}
 \]
 with a flat prior $Pr(T_1) = Pr(T_2)= Pr(T_3)$
\end{itemize}

All likelihood-based supports (aBayes, aLRT, SH-aLRT) amount to testing if $T_1$ is significantly better than $T_2$ and $T_3$. By focusing on one branch at the time rather than questioning the whole tree, likelihood-based supports are less conservative than $BP$ and $PP$. They can also recycle likelihood computed while estimating the focal tree and are therefore much faster to compute than standard $BP$. Finally, they proved to be accurate in simulations studies~\citep{Anisimova2011}. They are the default support values in PhyML~\citep{Guindon2003}. 

%%%%%%%%%%%%%%%%%%%%%%%%%%%%
\subsection{Outliers in the Data} \label{sec:outliers}

The aforementioned support values aggregate all variations in the data set and are unable to pinpoint variation due to outliers. The nature of resampling techniques is to use the empirical distribution as a surrogate for the true distribution. However, the empirical distribution may be polluted by outliers, defined here as ``entry in the data set that are anomalous with respect to
the behavior seen in the majority of the other entries in the data set''~\citep{Barnett1994}. This is a common occurrence in multi-locus studies where some characters can evolve according to one a tree, and others according to another tree~\citep{Degnan2009}. In that case, a single phylogeny is not a good fit to all the characters and~\citet{Swofford1996} argued that it should be interesting to pinpoint where the phylogeny is not a good fit of the molecular data. Restricting the analyses to congruent characters usually leads to higher support values~\citep{Bar-Hen2008}.
%%%%%%%%%%%%%%%%%%%%%%%%%%%%
% \subsubsection{Rogue Sites} \label{sec:rogue-sites}

Several approaches have been developed to identify outlier characters. Many studies \citep{Rodriguez-Ezpeleta2007, Burleigh2004} advocate removing fast-evolving characters which are a well-known cause of misleading phylogenetic signal and long branch attraction (LBA) where distantly related taxa are grouped together in the tree due to parallel or convergent evolution \citep{Felsenstein1978}. \citet{Lopez1999} also suggest to investigate and remove characters with high rate variations (\emph{i.e.} fast-evolving in some parts of the tree, slow-evolving in others). However both methods assume that good topologies are available to accurately estimate rates, leading to a circularity problem. 

\citet{Bar-Hen2008} adapted instead influence functions~\citep{Hampel1974} to phylogenetics in order to assess the impact of a single site on the likelihood. The main idea consists in removing one character at a time, to create \emph{jackknife} replicates, and to infer a tree on each replicate. Jackknife trees are used to find influential characters whose removal most affect the tree likelihood. \citet{Bar-Hen2008} report that influential sites have a strong impact on the topology and correspond mostly to fast evolving sites. All approaches found that removing outliers leads to more stable phylogenies but none is available as a routine in popular softwares. 

%%%%%%%%%%%%%%%%%%%%%%%%%%%%
\subsection{Taxon Sampling} \label{sec:rogue-taxa}

In phylogenomics studies, it is common to have conflicting trees with support values higher than $95$\% for all inner branches~\citep{Rydin2002}. This correspond to setups where the estimated tree has a very small estimation variance and differences between inferred trees result mostly from bias and modeling errors. In particular, \citet{Swofford1996} argues that adequate taxon sampling is one of the primary factors for accurate phylogenetic estimates, on par with enough sequence data. For example, dense taxon sampling can reduce the impact of LBA by splitting long branches. Similarly, \citet{Holland2003} and \citet{Shavit2007} showed that the inclusion of an outgroup to the analysis may disrupt the ingroup phylogeny. When there are only a few taxa, but many characters, phylogenetic analysis can produce high support values ($BP$, $PP$, etc.) for incorrect or misleading phylogenies~\citep{Rokas2003, Rokas2005, Heath2008}.

Analysis of sensitivity to taxon inclusion should be a part of careful and thorough phylogenetic analysis~\citep{Heath2008}. \citet{Mariadassou2012} defined a the Taxon Influence Index (TII) to assess the influence of each taxon on the phylogeny. Using any inference method, we define $T^*$ to be the tree inferred from the complete MSA. Let $T_k$ be a smaller tree, inferred from the
alignment deprived of taxon $k$ and $T_k^*$ the tree obtained by pruning taxon $k$ from $T^*$. 
The TII is the distance between trees $T_k$ and $T_k^*$, such that 
\[
TII(k) = d(T_k , T_k^*) 
\]
They found that most taxa have small TII(k) and little influence on the topology whereas a few are highly influential \emph{rogue taxa} and alter the phylogeny in clades even loosely related to their placement in the tree. \citet{Aberer2013} use a different approach to find rogue taxa, they start from a forest of trees (\emph{e.g.} bootstrap trees) and search for a small set of taxa whose pruning increases the agreement between trees in the forest. The method is implemented in the webservice RogueNarok. The rationale in both cases is that reliable trees over smaller taxa sets are preferable over uncertain trees of larger taxa sets. Both methods find that pruning rogue taxa improves accuracy and results in more stable phylogenies with higher support values. 

%%%%%%%%%%%%%%%%%%%%%%%%%%%%
%%%%%%%%%%%%%%%%%%%%%%%%%%%%
\section{Consensus Methods} \label{sec:consensus}

Boostrap, jackknife and bayesian estimation naturally produce a forest of trees with the same species set. But different trees can also be estimated by using different methods or different sources of data. One way to summarize the forest is to \emph{project} it on a focal tree to compute support values (see section~\ref{sec:robustness}). Alternatively, one can bypass the focal tree altogether and combine all trees in the forest to get a single tree. That is the purpose of consensus trees methods. 

%%%%%%%%%%%%%%%%%%%%%%%%%%%%
\subsection{Consensus Trees} \label{sec:consensus-tree}


\emph{Consensus trees} are trees that summarize a forest of trees with the same species set. We present here only the \emph{strict} consensus, the \emph{majority rule} consensus and the \emph{extended majority rule} consensus but there are many other consensus (see \citet{Bryant2003} for an extensive survey). The different notions are best understood on an example. Consider the forest featured in Figure~\ref{fig:consensus} with 40 copies of trees $T_1$, 48 of tree $T_2$ and 12 of tree $T_3$. Each tree is completely defined by the bipartitions it induces\footnote{or clades for for rooted trees}. For example, $T_1$ induces the partitions $AB|CDEF$, $ABCD|EF$ and $ABEF|CD$\footnote{or clades $AB$, $CD$, $EF$ and $CDEF$ if considered as rooted}, in addition to all trivial partitions $A|BCDEF$, $B|ACDEF$, etc not shown in the figure. Our three consensus methods scan the forest to build a list of all partitions occuring in the forest with their frequency of occurrence (middle column of Figure~\ref{fig:consensus}). They then select a subset of partitions according to some rules and build a consensus tree from that subset only. 

\begin{figure}
 \includegraphics[width=0.9\linewidth]{Figs/consensus}
 \caption{Left: a forest of $100$ trees, corresponding to $3$ topologies. Each colored cross correspond to a non trivial bipartition. Middle: Set of all bipartitions, with their occurence frequencies, found in the forest. Right: Different consensus made up by increasing large sets of partitions.}
 \label{fig:consensus}
\end{figure}

%%%%%%%%%%%%%%%%%%%%%%%%%%%%
\subsubsection{Strict Consensus}

The \emph{strict consensus} tree \citep{Rohlf1982} only uses partitions that appear in \textbf{all} trees, \emph{i.e.} with a 100\% occurence frequency. The strict consensus is fully compatible with all trees in the forest. However, it is less resolved than any tree and usually too strict. In our example, $T_1$ and $T_2$ and $T_3$ only differ in the position of $D$: if we removed $D$ from all trees, they would be identical. We could therefore expect a branch separating $EF$ from $ABC$. However, the set $CDEF$ is completely unresolved in the strict consensus. 

%%%%%%%%%%%%%%%%%%%%%%%%%%%%
\subsubsection{Majority-rule Consensus}

The \emph{majority-rule} consensus tree \citep{Margush1981} relaxes the condition that a bipartition must appear in all trees to be included in the consensus. Instead, it must only appear in \emph{most} trees, \emph{i.e.} have a occurence frequency higher than 50\%. Although not obvious, all such partitions are pairwise-compatible and can be used to build a proper tree \citep{Buneman1971}. The majority-rule tree is more resolved than the strict consensus one but is not compatible with all trees in the forest. In our example, the partition $ABCD|EF$ seen in the majority-rule consensus is in conflict with the partition $ABCF|DE$ present in $T_2$. 

%%%%%%%%%%%%%%%%%%%%%%%%%%%%
\subsubsection{Extended Majority-rule Consensus}

The \emph{extended majority-rule} consensus \citep{Felsenstein2005}, also called greedy consensus, relaxes the occurence frequency condition even further. The consensus is build by sequentially adding partition one at a time, in decreasing order of occurence and only if compatible with previously included partitions, until the tree is fully resolved or no more partitions can be added. Since all partitions with frequency higher than 50\% are compatible, they are part of the selection and the greedy consensus is thus a refinement of the majority-rule consensus. In our example, after including partitions $ABCD|EF$ and $AB|CDEF$, we can add either $ABC|DEF$ or $ABCF|DE$. The latter is in conflict with $ABCD|EF$ and thus not included whereas the former is compatible with both partitions and thus included. After addition of $ABC|DEF$, the greedy consensus is fully resolved. Note that the greedy consensus is different from all trees in the forest. 

%%%%%%%%%%%%%%%%%%%%%%%%%%%%
\subsubsection{Branch Lengths in Consensus Trees}

The strict, majority-rule and extended majority-rule consensus trees only use \emph{toplogies} and produce consensus topologies. One way to add branch lengths to that consensus is to take, for each branch, its average length over trees where it is present. This is the approach used in MrBayes \citep{Ronquist2003} when building a consensus tree from the sample of posterior trees. 

%%%%%%%%%%%%%%%%%%%%%%%%%%%%
\subsection{Distance-based Consensus} \label{sec:distance-based-consensus-tree}

The previous consensus methods are easy to understand and implement. Some also have a theoretical grounding that leads to generalizations. \citet{Barthelemy1986} showed that the majority-rule consensus is the \emph{center} of the forest in the sense that it minimizes the sum of Robinson-Foulds distances \citep{Robinson1979} to all trees in the forest. It is thus the forest \emph{median tree}. One could mininmize total squared distance to all trees in the forest to find the \emph{mean tree}. 

The Robinson-Foulds is but one of many distances between trees (see \citet{St.John2017} for an excellent review) and one can define other consensus similarly as the forest mean or median tree for some distance between tree. Although seducing, this approach suffered in the past from two shortcomings that severely limit its use. First, it was not obvious that the mean, or median, tree is well-defined or unique for some distances \citep{Billera2001}. Second, even when the mean was well-defined, there was no routine to compute it in practice. Recent progress in the field~\citep{Miller2015} have made it easier to compute sophisticated distances and use them to validate trees. % Computing the mean, variance and more general analyses of forest of trees based on tree distances is an active research field with many potential applications. 
 %3 pages %% MM

% %%%%%%%%%%%%%%%%%%%%%%%%%%%%
%%%%%%%%%%%%%%%%%%%%%%%%%%%%
\section{Testing} \label{sec:testing}

Likelihood based tests for 
\begin{itemize}
 \item topology against topology (or topologies)
 \item model against model
 \item node/branch against polytomy (=branch length)
\end{itemize}

%%%%%%%%%%%%%%%%%%%%%%%%%%%%
\subsection{Likelihood Based Tests} \label{sec:likelihood-tests}
KH, SH, AU

%%%%%%%%%%%%%%%%%%%%%%%%%%%%
\subsection{Model Testing} \label{sec:model-test}
model test

%%%%%%%%%%%%%%%%%%%%%%%%%%%%
\subsection{Node Testing} \label{sec:node-test}
Test polytomy using using nested models and LRT. Bayes factor in a Bayesian setting.

%%%%%%%%%%%%%%%%%%%%%%%%%%%%
%%%%%%%%%%%%%%%%%%%%%%%%%%%%
\section{Detection of conflicting signal} \label{sec:extensions}

Consensus trees assumes congruence among trees in the set. Trees that turn out to be far away from the consensus tree can be the sign of high variance or conflicting signals. While high variance implies weak phylogenetic signal, conflicting  signals implies incongruence. Most distance distances, such as RF, are discrete and lead to very coarse distance distributions. By contrast, continuous distances are a simple way to characterize variance and distinguish one from the other. Moreover continuous distance lend themselves nicely to standard aspects of inferential statistics such as confidence sets and hypothesis testing.

%Consider the space tree
%\begin{itemize}
% \item BHV space topology
% \item Means and variances in the tree space \textcolor{red}{briefly mentioned in previous section}
% \item Multivariate Analysis based on tree distances
% \end{itemize}

The  geometric  model of \cite{Billera2001} allows one to compare phylogenetic  trees,  with  the  same  leaf  set  of  cardinality $m$, in a quantitative way.  This  space  has  a  natural  metric, giving a way of measuring distance between phylogenetic trees and providing some procedures for averaging or combining several trees whose leaves are identical. This geometry also shows which trees appear within a fixed distance of a given tree and enables construction of convex hulls of a
set of trees. It also provides a justification for disregarding portions of a collection of trees that agree,  thus simplifying the space in which comparisons are to be made.

%%%%%%%%%%%%%%%%%%%%%%%%%%%%
\subsection{Tree Space definition} \label{sec:Tree-distances}
 The distance $d(T_i,T_j)$ between two trees $T_i$ and $T_j$ account for differences with respect to both their tree topologies (branching structure) and branch lengths. The space is constructed by representing each of the $(2m-3)!!$ possible tree topologies by a single non-negative Euclidean orthant of dimension $m-3$ (the largest possible number of internal branches). The orthants are then “glued together” along appropriate axes. Specifically, nearest neighbor interchange (NNI) topologies lie in adjacent non-negative orthants along the boundary corresponding to the collapse of the relevant NNI edge.

For two trees with different topologies, the BHV distance is the length of the shortest path between them that remains in the treespace.  The length of any path can be computed by calculating the Euclidean distance of the path restricted to each orthant that it passes though, and summing these lengths.  The shortest path is called a geodesic, and will pass from one orthant to the next orthant through lower-dimensional boundaries corresponding to trees with fewer splits. Since the space is nonpositively  curved, the geodesics are unique.

\begin{figure}
 \includegraphics{Figs/OrthantBHV}
 \caption{Orthants in the BHV space (one per topology) and geodesic paths between two trees corresponding to topologies $T$ and $T'$. Reproduced from \citet{Billera2001}.}
\end{figure}

In Euclidean space, the Fr\'echet mean is the point minimizing the sum of the squared distances to the sample points, and is equivalent to the coordinate-wise average of the sample points. The mean tree is not necessarily a refinement of the majority-rule consensus tree. Fr\'echet variance is the tree that minimize the sum of squared distances.  This variance is unique because treespace is non-positively curved. The Fr\'echet variance of a set of trees quantifies how spread out a set of trees is from their mean. See \cite{miller2015polyhedral,brown2017mean} for details.

%%%%%%%%%%%%%%%%%%%%%%%%%%%%
\subsection{Use of BHV distance} \label{sec:means-and-variance}

\cite{barden2017logarithm} proved a Central Limit Theorem on the BHV treespace, showing that the distribution of the sample means converges to a certain Gaussian distribution. It is useful for detecting splits of weak and strong support and in tree-valued hypothesis testing.

A key tool of \cite{barden2014limiting}is  the log map  that permits to  map  trees  from  their  metric  space  to  Euclidean space, where it is possible to model a tree estimate $\hat T$ as a noisy realization of the true tree $T$. Once  the model parameters are estimated, Euclidean multivariate analysis techniques to reduce the dimension of the trees can be used. This allows to visualize tree estimates, along with their uncertainties.

For example it is possible to use the variance covariance matrix to  estimate  the  principal  directions  of  variability  via principal components analysis. The axes of the $\mathbb{R}^m$ ellipsoid indicate the relative directions of precision, and the ellipsoid can be shrunk  to be wholly contained in the same orthant as $\hat T_n$. This gives an unambiguous indication of the relative
confidence in the edges of the estimated tree. Note that the procedure is unambiguous about the trees contained in the confidence set for a given confidence level $\alpha$ (see \cite{willis2016confidence}).

Recently, \cite{de2012phylo} developed a statistical non-parametric method to detect outlier trees from the set of gene trees. They first convert gene trees into vectors in a multidimensional Euclidean space and then apply multiple co-inertia analysis (MCOA)—an extension of principal coordinate analysis—directly to these vectorized gene trees. Their method, Phylo-MCOA, also detects outlier species, those whose position varies widely from tree to tree. Included in our results are simulation studies comparing our non-parametric method with Phylo-MCOA.

\cite{weyenberg2014kdetrees}  proposes  a non-parametric estimator of the distribution that generated the sample trees $T_1,\ldots,T_n$.  This estimator can be viewed as a refined version of histogram-based estimation of a density. The kernel function, is a non-negative function defined on pairs of trees, which measures how similar two trees are. Kernel density estimation use the fact that points close to sample points tend to have higher likelihood than distant outlier points.  The ultimate goal is to detect outlier trees, $T_j$, which are not actually drawn from the true distribution.
%%%%%%%%%%%%%%%%%%%%%%%%%%%%
%\subsection{Confidence Sets Based on other Distances} \label{sec:kernel}
%
%Kernel based methods (using a distance matrix) to find outlier and build confidence sets.

%%%%%%%%%%%%%%%%%%%%%%%%%%%%
\subsection{Conclusion}

Nowadays efficient algorithms permit to compute means of continuous distances in a reasonable time. Moreover a huge amount of work has been done to derive mathematical properties of theses distances. Therefore it is possible to go from a theoretical work to applications and one may expect that many important biological questions could be solved thanks to theses tools.

Genomic analysis leads to multiple trees du to different loci having different evolution history. Distances are useful tools to cluster trees and/or loci into congruent groups and performing separate inference on each group can lead to more robust phylogenetic estimates.
The strong interest in this methods will be certainly maintain  in the following years.
  %1 - 2 pages %% ABH

\bibliographystyle{plainnat}
\bibliography{robustness}

%%%%%%%%%%%%%%%%%%%%%%%%%%%%%%%%%%%%%%%%%%%%%%%%%%%%%%%%%%%%%%%%%%%%%%
%%%%%%%%%%%%%%%%%%%%%%%%%%%%%%%%%%%%%%%%%%%%%%%%%%%%%%%%%%%%%%%%%%%%%%
\end{document}
%%%%%%%%%%%%%%%%%%%%%%%%%%%%%%%%%%%%%%%%%%%%%%%%%%%%%%%%%%%%%%%%%%%%%%
%%%%%%%%%%%%%%%%%%%%%%%%%%%%%%%%%%%%%%%%%%%%%%%%%%%%%%%%%%%%%%%%%%%%%% 